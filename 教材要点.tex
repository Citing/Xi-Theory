\documentclass[a4paper, UTF8]{ctexart}
\usepackage{geometry}
\usepackage{fancyhdr}
\usepackage{setspace}
\usepackage{titlesec}
\usepackage[colorlinks, urlcolor=blue]{hyperref}

\geometry{top=2.6cm, bottom=2.6cm, left=2.45cm, right=2.45cm, headsep=0.4cm, foot=1.12cm}
\onehalfspacing

\pagestyle{fancy}
\lhead{教材要点}
\rhead{19-20秋冬}
\chead{习近平新时代中国特色社会主义思想概论}
\lfoot{}
\cfoot{\thepage}
\rfoot{}

\title{\huge{\heiti 习近平新时代中国特色社会主义思想概论\protect\\教材要点}}
\author{}
\date{}

\titleformat{\section}[block]{\LARGE\bfseries}{\arabic{section}}{1em}{}[]

\setlength{\headheight}{15pt}

\begin{document}
\maketitle

% 教材知识点
\section{毛泽东思想及其历史地位(了解历史即可)}
    \subsection{毛泽东思想的形成过程}
    \begin{itemize}
        \item 形成:第一次国内革命战争时期(1924-1927,国共合作,反对北洋军阀),分析中国社会各阶级在革命中的地位和作用,提出新民主主义革命的思想。
        土地革命战争(1927-1937)时期,农村包围城市、武装夺取政权,毛泽东思想初步形成。
        \item 成熟:遵义会议后,《实践论》《矛盾论》《论联合政府》,系统阐述新民主主义革命理论,1945年中共七大将毛泽东思想写入党章。
        \item 继续发展:解放战争时期和新中国成立后,提出人民民主专政理论,社会主义改造理论、关于严格区分和正确处理两类矛盾的学说。
    \end{itemize}

    \subsection{毛泽东思想的主要内容}
    \begin{itemize}
        \item 新民主主义革命理论:统一战线、武装斗争、党自身的建设
        \item 社会主义革命和社会主义建设理论:人民民主专政
        \item 革命军队建设和军事战略的理论:党指挥枪!
        \item 思想政治工作和文化工作的理论:百花齐放、百家争鸣
    \end{itemize}

    \subsection{毛泽东思想活的灵魂}
    \begin{itemize}
        \item 实事求是,一切从实际出发,理论联系实际
        \item 群众路线,一切为了群众,一切依靠群众,从群众中来,到群众中去,把党的正确主张变为群众的自觉行动
        \item 独立自主,走和平发展道路
    \end{itemize}

    \href{http://www.xinhuanet.com//politics/2013-12/26/c_118723453.htm}{请全文认真阅读本篇讲话}

    
\end{document}