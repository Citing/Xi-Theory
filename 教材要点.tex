\documentclass[a4paper, UTF8]{ctexart}
\usepackage{geometry}
\usepackage{fancyhdr}
\usepackage{setspace}
\usepackage{titlesec}
\usepackage[colorlinks, urlcolor=blue]{hyperref}

\geometry{top=2.6cm, bottom=2.6cm, left=2.45cm, right=2.45cm, headsep=0.4cm, foot=1.12cm}
\onehalfspacing

\pagestyle{fancy}
\lhead{教材要点}
\rhead{19-20秋冬}
\chead{习近平新时代中国特色社会主义思想概论}
\lfoot{}
\cfoot{\thepage}
\rfoot{}

\title{\huge{\heiti 习近平新时代中国特色社会主义思想概论\protect\\教材要点}}
\author{}
\date{}

\titleformat{\section}[block]{\LARGE\bfseries}{\arabic{section}}{1em}{}[]

\setlength{\headheight}{15pt}

\begin{document}
\maketitle

% 教材知识点
\section{毛泽东思想及其历史地位(了解历史即可)}
    \subsection{毛泽东思想的形成过程}
    \begin{itemize}
        \item 形成:第一次国内革命战争时期(1924-1927,国共合作,反对北洋军阀),分析中国社会各阶级在革命中的地位和作用,提出新民主主义革命的思想。
        土地革命战争(1927-1937)时期,农村包围城市、武装夺取政权,毛泽东思想初步形成。
        \item 成熟:遵义会议后,《实践论》《矛盾论》《论联合政府》,系统阐述新民主主义革命理论,1945年中共七大将毛泽东思想写入党章。
        \item 继续发展:解放战争时期和新中国成立后,提出人民民主专政理论,社会主义改造理论、关于严格区分和正确处理两类矛盾的学说。
    \end{itemize}

    \subsection{毛泽东思想的主要内容}
    \begin{itemize}
        \item 新民主主义革命理论:统一战线、武装斗争、党自身的建设
        \item 社会主义革命和社会主义建设理论:人民民主专政
        \item 革命军队建设和军事战略的理论:党指挥枪!
        \item 思想政治工作和文化工作的理论:百花齐放、百家争鸣
    \end{itemize}

    \subsection{毛泽东思想活的灵魂}
    \begin{itemize}
        \item 实事求是,一切从实际出发,理论联系实际
        \item 群众路线,一切为了群众,一切依靠群众,从群众中来,到群众中去,把党的正确主张变为群众的自觉行动
        \item 独立自主,坚定不移走中国特色社会主义道路,走和平发展道路
    \end{itemize}

    \href{http://www.xinhuanet.com//politics/2013-12/26/c_118723453.htm}{请全文认真阅读本篇讲话}

\section{邓小平理论(了解几次会议的内容)}
    \begin{itemize}
        \item 十一届三中全会:重新确立了解放思想,实事求是的思想路线
        \item 十二大:建设有中国特色的社会主义
        \item 十二届三中全会:提出社会主义经济是公有制基础上的有计划的商品经济
        \item 十三大:系统地论述了我国社会主义初级阶段理论,明确“一个中心、两个基本点”
        \item 1989年
        \item 1992年南巡:三个有利于,两手都要硬
        \item 十五大:邓小平理论写入党章
    \end{itemize}

\section{“三个代表”重要思想}
    他改变了中国

\section{科学发展观}
    涛声依旧

\section{习近平新时代中国特色社会主义思想及其历史地位}
    \textbf{习近平新时代中国特色社会主义思想是党和国家必须长期坚持的指导思想}

    \emph{这些话背下来可用于每道题}
    
    中国共产党第十九次全国代表大会,把习近平新时代中国特色社会主义思想确立为党必须长期坚持的指导思想并庄严写入党章,实现了党的指导思想的与时俱进。
    这是一个历史性决策和历史性贡献,体现了党在政治上理论上的高度成熟、高度自信。
    第十三届全国人民代表大会第一次会议通过的宪法修正案,郑重地把习近平新时代中国特色社会主义思想载入宪法,实现了国家指导思想的与时俱进,反映了全国各族人民共同意志和全社会共同意愿。
    习近平新时代中国特色社会主义思想,是新时代中国共产党的思想旗帜,是国家政治生活和社会生活的根本指针,是当代中国马克思主义、二十一世纪马克思主义。

    世界怎么了?应该怎么办?
    在这样大发展大变革大调整的背景下,以习近平同志为核心的党中央,为解决世界经济、国际安全、全球治理等一系列重大问题提供了新的方向、新的方案、新的选择。
    中国发展理念、发展道路、发展模式的影响力、吸引力显著增强,中国日益发挥着世界和平建设者、全球发展贡献者、国际秩序维护着的重要作用,前所未有地走进世界舞台中央。
    习近平新时代中国特色社会主义思想,正是在把握世界发展大势、应对全球共同挑战、维护人类共同利益的过程中创立并不断丰富发展的。

    党的十八大以来,党和国家事业之所以取得全方位、开创性历史成就,发生深层次、根本性历史变革,根本在于以习近平同志为核心的党中央的坚强领导,根本在于习近平新时代中国特色社会主义思想的科学指导。

    习近平新时代中国特色社会主义思想,坚守中国共产党人为人民谋幸福的初心,坚持人民主体地位,坚持一切为了人民、一切依靠人民,彰显了人民是历史的创造者、人民是真正英雄的唯物史观,彰显了以人为本、人民至上的价值取向,彰显了立党为公、执政为民的执政理念。
    这一思想承载中国共产党人为民族某复兴的使命,擘画实现民族复兴中国梦的宏伟蓝图,高扬中华民族伟大创造精神、伟大奋斗精神、伟大团结精神、伟大梦想精神,传承和弘扬中华优秀传统文化,为实现中华民族伟大复兴提供了强大精神力量。
    这一思想担当中国共产党人为世界某大同的责任,饱含对人类发展重大问题的睿智思考和独特创见,洞察时代风云,把握时代脉搏,引领时代潮流,为应对全球共同挑战、共同问题提供了中国智慧和中国方案,为推动构建人类命运共同体、维护人类共同利益和共同价值作出了重要贡献。

    习近平新时代中国特色社会主义思想的核心内容是“八个明确”和“十四个坚持”。

    \subsection{八个明确}
    \begin{itemize}
        \item 明确坚持和发展中国特色社会主义,\textbf{总任务}是实现社会主义现代化和中华民族伟大复兴,在全面建成小康社会的基础上,分两步走在本世纪中叶建成富强民主文明和谐美丽的社会主义现代化强国;
        \item 明确新时代我国社会\textbf{主要矛盾}是人民日益增长的美好生活需要和不平衡不充分的发展之间的矛盾,必须坚持以人民为中心的发展思想,不断促进人的全面发展、全体人民共同富裕;
        \item 明确中国特色社会主义事业\textbf{总体布局}是“五位一体”、\textbf{战略布局}是“四个全面”,强调坚定道路自信、理论自信、制度自信、文化自信;
        \item 强调\textbf{全面深化改革总目标}是完善和发展中国特色社会主义制度、推进国家治理体系和治理能力现代化;
        \item 明确\textbf{全面依法治国总目标}是建设中国特色社会主义法治体系、建设社会主义法治国家;
        \item 明确\textbf{党在新时代的强军目标}是建设一支听党指挥、能打胜仗、作风优良的人民军队,把人民军队建设成为世界一流军队;
        \item 明确\textbf{中国特色大国外交}要推动构建新型国际关系,推动构建人类命运共同体;
        \item 明确中国特色社会主义\textbf{最本质的特征}是中国共产党领导,中国特色社会主义制度的最大优势是中国共产党领导,党是最高政治领导力量,提出新时代党的建设总要求,突出政治建设在党的建设中的重要地位。
    \end{itemize}

    \subsection{十四个坚持}
    \begin{itemize}
        \item 坚持党对一切工作的领导
        \item 坚持以人民为中心
        \item 坚持全面深化改革
        \item 坚持新发展理念
        \item 坚持人民当家作主
        \item 坚持全面依法治国
        \item 坚持社会主义核心价值体系
        \item 坚持在发展中保障和改善民生
        \item 坚持人与自然和谐共生
        \item 坚持总体国家安全观
        \item 坚持党对人民军队的绝对领导
        \item 坚持“一国两制”和推进祖国统一
        \item 坚持推动构建人类命运共同体
        \item 坚持全面从严治党
    \end{itemize}

\section{坚持和发展中国特色社会主义的总任务}
    \textbf{坚持和发展中国特色社会主义,总任务是实现社会主义现代化和中华民族伟大复兴}

    \subsection{中国梦的本质是国家富强、民族振兴、人民幸福}
    实现中国民族伟大复兴的中国梦,就是要实现国家富强、民族振兴、人民幸福。这既深深体现了今天中国人的理想,也深深反映了中国人自古以来不懈追求进步的光荣传统。
    国家富强,就是要全面建成小康社会,并在此基础上建设富强民主文明和谐美丽的社会主义现代化强国;
    民族振兴,就是要使中华民族更加坚强有力地自立于世界民族之林,为人类作出新的更大的贡献;
    人民幸福,就是要坚持以人民为中心,增进人民福祉,促进人的全面发展,朝着共同富裕方向稳步前进。
    中国梦把国家的追求、民族的向往、人民的期盼融为一体,体现了中华民族和中国人民的整体利益,表达了每一个中华儿女的共同愿景,已成为激荡在近十四亿人心中的高昂旋律,成为中华民族团结奋斗的最大公约数和最大同心圆。

    中国梦归根到底是人民的梦,必须紧紧依靠人民来实现,必须不断为人民造福。人民是中国梦的主体,是中国梦的创造者和享有者。
    习近平总书记强调:“中国梦不是镜中花、水中月,不是空洞的口号,其最深沉的根基在中国人民心中。”

    中国梦是中国人民追求幸福的梦,也同世界人民的梦想息息相通。
    中国一心一意办好自己的事情,实现国家发展和稳定,既是对自己负责,也是为世界作贡献。
    中国人民深知,中国发展得益于国际社会,愿意同各国人民在实现各自梦想的过程中相互支持、相互帮助。
    中国将同国际社会一道,推动实现持久和平、共同繁荣的世界梦,为人类和平与发展的崇高事业作出新的更大的贡献。

    \subsection{实现伟大梦想必须进行伟大斗争、建设伟大工程、推进伟大事业}

    实现伟大梦想,必须进行伟大斗争。
    社会是在矛盾运动中前进的,有矛盾就会有斗争。
    我们党要团结带领人民有效应对重大挑战、抵御重大风险、克服重大阻力、化解重大矛盾、解决重大问题,必须进行具有许多新的历史特点的伟大斗争。
    要牢牢掌握斗争主动权,发扬斗争精神、增强斗争本领,敢于斗争、善于斗争,在事关中国特色社会主义前途命运的大是大非问题上坚定不移,在改革发展稳定工作中敢于碰硬,在全面从严治党上敢于动硬,在维护国家核心利益上敢于针锋相对,不在困难面前低头,不在挑战面前退缩,不拿原则做交易,不在任何压力下吞下损害中华民族根本利益的苦果。
    充分认识这场伟大斗争的长期性、复杂性、艰巨性,到重大斗争一线去真枪真刀磨砺,以“踏平坎坷成大道,斗罢艰险又出发”的顽强意志,不断夺取伟大斗争新胜利。

    实现伟大梦想,必须建设伟大工程。
    这个伟大工程就是我们党正在深入推进的党的建设新的伟大工程。
    历史已经并将继续证明,没有中国共产党的领导,民族复兴必然是空想。
    
    实现伟大梦想,必须推进伟大事业。中国特色社会主义是改革开放以来党的全部理论和实践的主题。

    \subsection{分两步走全面建成社会主义现代化强国}
    第一个阶段,从2020年到2035年,在全面建成小康社会的基础上,再奋斗十五年,基本实现社会主义现代化。
    第二个阶段,从2035年到本世纪中叶,在基本实现现代化的基础上,再奋斗十五年,把我国建成富强民主文明和谐美丽的社会主义现代化强国。

    \subsection{全面用好我国发展的重要战略机遇期}
    习近平总书记深刻指出:“我国发展仍处于并将长期处于重要战略机遇期。”

    从\textbf{经济}方面看,有五个新机遇:
    一是加快经济结构优化升级带来新机遇。
    二是提升科技创新能力带来新机遇。
    三是深化改革开放带来新机遇。
    四是加快绿色发展带来新机遇。
    五是参与全球经济治理体系变革带来新机遇。

\section{新时代中国特色社会主义经济建设}
    \textbf{以新发展理念引领经济高质量发展}

    \subsection{新发展理念}
    创新是引领发展的第一动力,创新发展注重的是解决发展动力问题,必须把创新摆在国家发展全局的核心位置,让创新贯穿党和国家一切工作。

    协调是持续健康发展的内在要求,协调发展注重的是解决发展不平衡问题,必须正确处理发展中的重大关系,不断增强发展整体性。

    绿色是永续发展的必要条件和人民对美好生活追求的重要体现,绿色发展注重的是解决人与自然和谐共生问题,必须实现经济社会发展和生态环境保护协同共进,为人民群众创造良好生产生活环境。

    开放是国家繁荣发展的必由之路,开放发展注重的是解决发展内外联动问题,必须发展更高层次的开放型经济,以扩大开放推进改革发展。

    共享是中国特色社会主义的本质要求,共享发展注重的是解决社会公平正义问题,必须坚持全民共享、全面共享、共建共享、渐进共享,不断推进全体人民共同富裕。

    新发展理念具有很强的战略性、纲领性、引领性,必须贯穿经济活动全过程。
    要努力提高统筹贯彻新发展理念的能力和水平,把新发展理念作为指挥棒、红绿灯,对不适应、不适合甚至违背新发展理念的认识要立即调整,行为要坚决纠正,做法要彻底摒弃,真正做到崇尚创新、注重协调、倡导绿色、厚植开放、推进共享。

    \subsection{我国经济由高速增长转向高质量发展}
    党的十八大以来,我们对经济发展阶段性特征的认识不断深化。
    2013年,党中央作出判断,我国经济发展正处于增长速度换挡期、结构调整阵痛期和前期刺激政策消化期“三期叠加”阶段。
    2014年,提出我国经济发展进入新常态。
    在新常态下,增长速度要从高速转向中高速,发展方式要从规模速度型转向质量效率型,经济结构调整要从增量扩能为主转向调整存量、做优增量并举,发展动力要从主要依靠资源和低成本劳动力等要素投入转向创新驱动。
    党的十九大进一步明确提出,我国经济已由高速增长阶段转向高质量发展阶段。

    \subsection{把推进供给侧结构性改革作为主线}
    我国经济运行面临的突出矛盾和问题,虽然有周期性、总量性因素,但根源是重大结构性失衡。
    主要表现为“三大失衡”,即实体经济结构性供需失衡、金融和实体经济失衡、房地产和实体经济失衡。

    深化供给侧结构性改革、推动经济高质量发展,总的要求是“巩固、增强、提升、畅通”八字方针,巩固“三去一降一补”(去产能、去杠杆、去库存、降成本、补短板)成果,推动更多产能过剩行业加快出清,降低全社会各类营商成本,加大基础设施等领域补短板力度。
    增强微观主体活力,发挥企业和企业家主观能动性,建立公平开放透明的市场规则和法治化营商环境,促进正向激励和优胜劣汰,发展更多优质企业。
    提升产业链水平,注重利用技术创新和规模效应形成新的竞争优势,培育和发展新的产业集群。
    畅通国民经济循环,加快建设统一开放、竞争有序的现代市场体系,提高金融体系服务实体经济能力,形成国内市场和生产主体、经济增长和就业扩大、金融和实体经济良性循环。

    \subsection{建设现代化经济体系}
    要建设创新引领、协同发展的产业体系,统一开放、竞争有序的市场体系,体现效率、促进公平的收入分配体系,彰显优势、协调联动的城乡区域发展体系,资源节约、环境友好的绿色发展体系,多元平衡、安全高效的全面开放体系,充分发挥市场作用、更好发挥政府作用的经济体制。

    大力发展实体经济,筑牢现代化经济体系的坚实基础;
    加快实施创新驱动发展战略,强化现代化经济体系的战略支撑;
    积极推动城乡区域协调发展,优化现代化经济体系的空间布局;
    着力发展开放型经济,提高现代化经济体系的国际竞争力;
    深化经济体制改革,完善现代化经济体系的制度保障。

\section{新时代中国特色社会主义政治建设}
    \subsection{发展社会主义民主政治}
    走中国特色社会主义政治发展道路,必须坚持党的领导、人民当家作主、依法治国有机统一。
    党的领导是人民当家作主和依法治国的根本保证,人民当家作主是社会主义民主政治的本质特征,依法治国是党领导人民治理国家的基本方式。

    我国实行工人阶级领导的、以工农联盟为基础的人民民主专政的国体,实行人民代表大会制度的政体,实行中国共产党领导的多党合作和政治协商制度,实行民族区域自治制度,实行基层群众自治制度。

    协商民主是在中国共产党领导下,人民内部各方面围绕改革发展稳定重大问题和涉及群众利益的实际问题,在决策之前和决策实施中,开展广泛协商,努力形成共识的重要民主形式。
    协商民主是中国社会主义民主政治的特有形式和独特优势,是实现党的领导的重要方式,丰富了民主的形式,拓展了民主的渠道,丰富了民主的内涵。
    发展协商民主,必须推进协商民主广泛、多层、制度化发展,统筹推进政党协商、人大协商、政府协商、政协协商、人民团体协商、基层协商及社会组织协商,加强协商民主制度建设,形成完整的制度程序和参与实践,保证人民在日常政治生活中有广泛持续深入参与的权利。

    \subsection{坚持“一国两制”和推进祖国统一}
    中央贯彻“一国两制”方针坚持两点,一是坚定不移,不会变、不会动摇;二是全面准确,确保“一国两制”在香港、澳门的实践不走样、不变形,始终沿着正确方向前进。

    要严格依照宪法和基本法办事。在落实宪法和基本法确定的宪制秩序时,把中央依法行使权力和特别行政区履行主体责任有机结合起来,完善与基本法实施相关的制度和机制,坚决维护宪法和基本法的权威。

    “和平统一、一国两制”是解决台湾问题的基本方针,也是实现国家统一的最佳方式。

    “九二共识”:海峡两岸同属一个中国,共同努力谋求国家统一。

\section{新时代中国特色社会主义文化建设}
    \textbf{推动社会主义文化繁荣}
    
    \subsection{建设具有强大凝聚力和引领力的社会主义意识形态}
    必须高举马克思主义、中国特色社会主义伟大旗帜,巩固马克思主义在意识形态领域的指导地位,巩固全党全国人民团结奋斗的共同思想基础,建设具有强大凝聚力和引领力的社会主义意识形态,建设具有强大生命力和创造力的社会主义精神文明,建设具有强大感召力和影响力的中华文化软实力。

    党和国家指导思想在我国社会主义意识形态中占据统摄地位,必须持续加强理论武装工作。
    把坚定“四个自信”作为建设社会主义意识形态的关键,宣传好社会主义现代化建设辉煌成就,讲清楚成就背后的理论逻辑、制度原因,增强干部群众对中国特色社会主义的信心和底气。

    做好意识形态工作,必须坚持和加强党对意识形态工作的全面领导。牢牢掌握意识形态工作领导权,坚持以立为本、立破并举,推进社会主义意识形态建设,使全体人民在理想信念、价值理念、道德观念上紧紧团结在一起。

    \subsection{用社会主义核心价值观凝心聚力}
    富强民主文明和谐自由平等公正法治爱国敬业诚信友善

    \subsection{推动中华优秀传统文化创造性转化、创新性发展}
    传承和弘扬中华优秀传统文化,要重点做好创造性转化和创新性发展,使之与现实文化相融相通。

    \subsection{进行无愧于时代的文艺创造}
    要把满足人民精神文化需求作为文艺和文艺工作的出发点和落脚点,把人民作为文艺表现的主体,把人民作为文艺审美的鉴赏家和评判者,把为人民服务作为文艺工作者的天职。

    \subsection{营造风清气正的网络空间}
    要坚持正能量是总要求,管得住是硬道理,用得好是真本事,科学认识网络传播规律,提高用网治网水平,推动互联网这个最大变量变成事业发展的最大增量。
    加强网络空间治理,构建良好网络秩序。

    \subsection{提高国家文化软实力}
    讲好中国故事是树立当代中国良好形象、提升国家文化软实力的重要战略任务。

\section{新时代中国特色社会主义社会建设}
    \textbf{带领人民创造更加幸福美好生活}

    \subsection{增进民生福祉是发展的根本目的}
    坚持在发展中保障和改善民生。必须抓住人民最关心最直接最现实的利益问题,抓住最需要关心的人群,在更高水平上实现幼有所育、学有所教、劳有所得、病有所医、老有所养、住有所居、弱有所扶,让人民有更多、更直接、更实在的获得感、幸福感、安全感。

    必须把教育事业放在优先位置,深化教育改革,加快教育现代化,建设教育强国,办好人民满意的教育。

    要把稳就业摆在突出位置,实施就业优先政策,实现更高质量和更充分就业。

    要坚持按劳分配原则,完善按要素分配的体制机制,促进收入分配更合理、更有序。

    要按照兜底线、织密网、建机制的要求,全面建成覆盖全民、城乡统筹、权责清晰、保障适度、可持续的多层次社会保障体系。

    要深化医药卫生体制改革,健全现代医院管理制度。

    \subsection{坚决打赢脱贫攻坚战}
    党的十八大以来,党中央实施精准扶贫、精准脱贫,加大扶贫投入,创新扶贫方式,扶贫开发工作呈现新局面。

    实施“五个一批”工程,即发展生产脱贫一批,易地搬迁脱贫一批,生态补偿脱贫一批,发展教育脱贫一批,社会保障兜底一批。

    稳定实现贫困人口“两不愁三保障”,即不愁吃不愁穿,义务教育、基本医疗、住房安全有保障。

    \subsection{总体国家安全观}
    坚持总体国家安全观,必须坚持国家利益至上,以人民安全为宗旨,以政治安全为根本,以经济安全为基础,以军事、文化、社会安全为保障,以促进国际安全为依托,维护各领域国家安全,构建国家安全体系,走中国特色国家安全道路。

    维护重点领域国家安全是主阵地、主战场。全面贯彻落实总体国家安全观,要聚焦重点,抓纲带目,把确保政治安全作为首要任务,统筹推进各重点领域国家安全工作。

\section{新时代中国特色社会主义生态文明建设}
    \textbf{建设美丽中国}

    \subsection{坚持人与自然和谐共生}
    绿水青山就是金山银山,阐述了经济发展和生态环境保护的关系,解释了保护生态环境就是保护生产力、改善生态环境就是发展生产力的道理,指明了实现发展和保护协同共生的新路径。
    生态环境保护和经济发展不是矛盾对立的关系,而是辩证统一的关系。生态环境保护的成败归根到底取决于经济结构和经济发展方式。
    在发展中保护,在保护中发展。

    \subsection{推动形成绿色发展方式和生活方式}
    加快形成绿色发展方式,重点是调整经济结构和能源结构,优化国土空间开发布局,培育壮大节能环保产业、清洁生产产业、清洁能源产业,推进生产系统和生活系统循环链接。

    加快形成绿色生活方式,要在全社会牢固树立生态文明理念,增强全民节约意识、环保意识、生态意识,培养生态道德和行为习惯,让天蓝地绿水清深入人心。

    统筹山水林田湖草系统治理。

    实现最严格的生态环境保护制度。

\section{“四个全面”战略布局}
    \subsection{全面建成小康社会}
    决胜全面建成小康社会,要紧扣我国社会主要矛盾变化,统筹推进经济建设、政治建设、文化建设、社会建设、生态文明建设,坚定实施科教兴国战略、人才强国战略、创新驱动发展战略、乡村振兴战略、区域些天发展战略、可持续发展战略、军民融合发展战略,
    坚决打好防范重大风险、精准脱贫、污染防治的攻坚战,使全面建成小康社会得到人民认可、经得起历史检验。

    \subsection{全面深化改革}
    \textbf{新时代坚持和发展中国特色社会主义的根本动力}

    全面深化改革的总目标是完善和发展中国特色社会主义制度,推进国家治理体系和治理能力现代化。

    处理好解放思想和实事求是的关系,处理好顶层设计和摸着石头过河的关系,处理好整体推进和重点突破的关系,处理好胆子要大、步子要稳的关系,处理好改革、发展、稳定的关系。

    \subsection{全面依法治国}
    \textbf{新时代坚持和发展中国特色社会主义的本质要求}

    中国特色社会主义法治道路的核心要义,就是要坚持党的领导,坚持中国特色社会主义制度,贯彻中国特色社会主义法治理论。
    党的领导是中国特色社会主义最本质的特征,是社会主义法治最根本的保证。
    坚持中国特色社会主义法治道路,最根本的是坚持党的领导。
    中国特色社会主义制度是中国特色社会主义法治体系的根本制度基础,是全面推进依法治国的根本制度保障。
    中国特色社会主义法治理论是中国特色社会主义法治体系的理论指导和学理支撑,是全面推进依法治国的行动指南。
    这三个方面规定和确保了中国特色社会主义法治体系的制度属性和前进方向。

    人民是依法治国的主体和力量源泉。平等是社会主义法律的基本属性,是社会主义法治的基本要求。

    \subsection{全面从严治党}
    “坚持党要管党、全面从严治党”是新时代党的建设的根本方针。

    党的十九大首次把党的政治建设纳入党的建设总体布局,并强调“以党的政治建设为统领”“把党的政治建设摆在首位”。

    “不忘初心、牢记使命”主题教育,总要求“守初心、担使命,找差距、抓落实”。

\section{新时代国防和军队建设}

\section{新时代中国特色大国外交}

\end{document}