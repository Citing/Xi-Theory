\documentclass[a4paper, UTF8]{ctexart}
\usepackage{geometry}
\usepackage{fancyhdr}
\usepackage{setspace}
\usepackage{titlesec}
\usepackage[colorlinks, urlcolor=blue]{hyperref}

\geometry{top=2.6cm, bottom=2.6cm, left=2.45cm, right=2.45cm, headsep=0.4cm, foot=1.12cm}
\onehalfspacing

\pagestyle{fancy}
\lhead{四中全会}
\rhead{19-20秋冬}
\chead{习近平新时代中国特色社会主义思想概论}
\lfoot{}
\cfoot{\thepage}
\rfoot{}

\title{\huge{\heiti 中国共产党第十九届中央委员会\protect\\第四次全体会议\protect}}
\author{}
\date{}

\titleformat{\section}[block]{\LARGE\bfseries}{\arabic{section}}{1em}{}[]

\setlength{\headheight}{15pt}

\begin{document}
\maketitle

% 四中全会
\section{会议信息}
    \begin{itemize}
        \item 时间:2019年10月28日至31日
        \item 文件:《中共中央关于坚持和完善中国特色社会主义制度、推进国家治理体系和治理能力现代化若干重大问题的决定》
    \end{itemize}
    
\subsection{重要判断}
    中国特色社会主义制度是党和人民在长期实践探索中形成的科学制度体系,我国国家治理一切工作和活动都依照中国特色社会主义制度展开,我国国家治理体系和治理能力是中国特色社会主义制度及其执行能力的集中体现。

\subsection{重要结论}
    实践证明,中国特色社会主义制度和国家治理体系是以马克思主义为指导、植根中国大地、具有深厚中华文化根基、深得人民拥护的制度和治理体系,是具有强大生命力和巨大优越性的制度和治理体系,是能够持续推动拥有近十四亿人口大国进步和发展、确保拥有五千多年文明史的中华民族实现“两个一百年”奋斗目标进而实现伟大复兴的制度和治理体系。

\subsection{总体目标}
    坚持和完善中国特色社会主义制度、推进国家治理体系和治理能力现代化的总体目标是,到我们党成立一百年时,在各方面制度更加成熟更加定型上取得明显成效;到二〇三五年,各方面制度更加完善,基本实现国家治理体系和治理能力现代化;到新中国成立一百年时,全面实现国家治理体系和治理能力现代化,使中国特色社会主义制度更加巩固、优越性充分展现。

\subsection{13个坚持和完善}
    \begin{itemize}
        \item 坚持和完善党的领导制度体系,提高党科学执政、民主执政、依法执政水平。
        \item 坚持和完善人民当家作主制度体系,发展社会主义民主政治。
        \item 坚持和完善中国特色社会主义法治体系,提高党依法治国、依法执政能力。
        \item 坚持和完善中国特色社会主义行政体制,构建职责明确、依法行政的政府治理体系。
        \item 坚持和完善社会主义基本经济制度,推动经济高质量发展。
        \item 坚持和完善繁荣发展社会主义先进文化的制度,巩固全体人民团结奋斗的共同思想基础。
        \item 坚持和完善统筹城乡的民生保障制度,满足人民日益增长的美好生活需要。
        \item 坚持和完善共建共治共享的社会治理制度,保持社会稳定、维护国家安全。
        \item 坚持和完善生态文明制度体系,促进人与自然和谐共生。生态文明建设是关系中华民族永续发展的千年大计。
        \item 坚持和完善党对人民军队的绝对领导制度,确保人民军队忠实履行新时代使命任务。
        \item 坚持和完善“一国两制”制度体系,推进祖国和平统一。
        \item 坚持和完善独立自主的和平外交政策,推动构建人类命运共同体。
        \item 坚持和完善党和国家监督体系,强化对权力运行的制约和监督。
    \end{itemize}
    
\end{document}